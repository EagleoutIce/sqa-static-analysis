\documentclass[
   aspectratio=169, % default is 43
   10pt, % font size, default is 11pt
   nosectionframes,
   uniqueslidenumber
]{beamer}

\usepackage[T1]{fontenc}
\usepackage[utf8]{inputenc}
\usepackage[sfdefault]{FiraSans}
\usepackage{FiraMono}
\usepackage{tikz}
\usetikzlibrary{arrows.meta}

\usepackage{../slide-template-uulm/fancybeamer} % use the fancy beamer package
\usepackage{../slide-template-uulm/fancyuulm}
\setpaths{{../slide-template-uulm/}{../slide-template-uulm/logos/}{../slide-template-uulm/empty-slides/}{../xlistings/}}

\usepackage{../code-animation/code-animation}
\usepackage[fakeminted,print]{../xlistings/xlistings}
\xlstsetmintedstyle{plain}
\LoadLanguages{R,Java}
\lstcolorlet{keywordA}{black!70!red}%
\lstcolorlet{keywordB}{black!70!red}%
\lstcolorlet{keywordC}{black!70!red}%
\lstcolorlet{numbers}{black!70!yellow}%

% bib
\usepackage[style=alphabetic,backend=biber]{biblatex}
\addbibresource{./references.bib}

\usepackage{stmaryrd}

\RequirePackage{mdsymbol}
\def\LeftArrow{\text{\BeginAccSupp{method=escape,ActualText={<-}}\(\leftarrow\)\EndAccSupp{}}}
\def\RightArrow{\text{\BeginAccSupp{method=escape,ActualText={->}}\(\rightarrow\)\EndAccSupp{}}}
\def\DoubleLeftArrow{\text{\BeginAccSupp{method=escape,ActualText={<<-}}\(\twoheadleftarrow\)\EndAccSupp{}}}
\def\DoubleRightArrow{\text{\BeginAccSupp{method=escape,ActualText={->>}}\(\twoheadrightarrow\)\EndAccSupp{}}}
\lstset{add to literate={<-}{{{\!\LeftArrow\!}}}2
   {<<-}{{{\!\DoubleLeftArrow\!}}}2
   {->}{{{\!\RightArrow\!}}}2
   {->>}{{{\!\DoubleRightArrow\!}}}2
}
\makeatletter
\def\ca@Strut{%
    \vphantom{%
        \xlst@font@fs@normal\xlst@styles@lst@basic\strut
    }%
}

\def\AbstractInfo#1{\ensuremath{\mathcolor{gray}{\Lbag}\,#1\,\mathcolor{gray}{\Rbag}}}
\def\Set#1{\ensuremath{\{#1\kern1pt\}}}
\def\IntCC#1#2{\ensuremath{[#1\,..\,#2]}}
\def\IntOC#1#2{\ensuremath{(#1\,..\,#2]}}
\def\IntCO#1#2{\ensuremath{[#1\,..\,#2)}}
\def\IntOO#1#2{\ensuremath{(#1\,..\,#2)}}

\title[Abstract Interpretation]{Abstract Interpretation}
\subtitle[SQA]{Software Quality Assurance - Static Code Analysis, II}
\author[F. Sihler]{Florian Sihler}
\date{\today} % use a particular date here if needed

\fancylogos{sp,uulm} % define logos that are spread evenly across the bottom of the title slide

\begin{document}

\maketitle[titleimage/title]

\section{The Why}
\begin{frame}[fragile]{\insertsection}
\AnimateCode{onslide={o2:{3,...,7},-,-,-,-,-,-},handout=2/1,first slide=2}
\begin{minted}[escapeinside=||,lineskip=1pt]{java}
public static void main(String[] args) {
    int a = 1; |\tikzmarknode{@a1}{\strut}|
    double r = Math.random() * 10; |\tikzmarknode{@r1}{\strut}|
    if (r > 5) { |\tikzmarknode{@r2}{\strut}|
       a = 2; |\tikzmarknode{@a2}{\strut}|
    }
    System.out.println(a); |\tikzmarknode{@a3}{\strut}|
}
\end{minted}
\endAnimateCode
\begin{tikzpicture}[overlay,remember picture]
   \onslide<3->{%
      \coordinate (@a1) at (pic cs:@a1);
      \node[right=4cm] (@a1) at (@a1) {\AbstractInfo{a \in \Set{1}}};
   }%
   \onslide<4->{%
      \coordinate (@r1) at (pic cs:@r1);
      \node[right] at (@r1-|@a1.west) {\AbstractInfo{r \in \IntCO{0}{10}}};
   }%
   \onslide<5->{%
      \coordinate (@a2) at (pic cs:@a2);
      \node[right] at (@a2-|@a1.west) {\AbstractInfo{a \in \Set{2}}};
   }%
   \onslide<6->{%
      \coordinate (@a3) at (pic cs:@a3);
      \node[right] (@set) at (@a3-|@a1.west) {\AbstractInfo{a \in \Set{1, 2}}};
      \onslide<7->{%
         \node[right] at (@set.east) {\(\to\)~\;Valid? Ok? Safe?};
      }
   }
\end{tikzpicture}
\begin{itemize}
   \item<8-> We want to proof, that a program satisfies certain properties
\end{itemize}
\end{frame}

\subsection{Origins}
\def\Mixin{}%
\tikzset{%
   history-line/.style={line width=1.85mm,gray!30!white\Mixin,line cap=round,rounded corners=2pt},
   history-line skip/.style={history-line, line width=.75mm,loosely dotted},
   history-event/.style={history-line,gray\Mixin,line width=1mm,{Circle[length=1.85mm]}-,shorten <= -1.85mm/2},
   history@box/.style={yshift=.675\baselineskip,black\Mixin,text width=5.75cm,font=\small},
   history-range/.style={history-line, gray!60!white\Mixin,line cap=round,-{Triangle Cap}}   
}

% #1 left/right
% #2 when
% #3 what
% #4 optional comment
\def\historybox#1#2#3#4{node[history@box,below #1] (@) {\textbf{#2}: #3\ifx!#4!\else\\\footnotesize\itshape#4\par\fi}}
\begin{frame}[c]{\insertsubsection}
\centering\vspace*{-11.5mm}\begin{tikzpicture}
   \only<-4|handout:1>{\draw[history-line] (-.33,-.33) -- ++(.33,.33) -- ++(2,0) coordinate (@2)++(1,0) -- ++(9,0) node[above left,gray] {Static Analysis };}
   \onslide<5|handout:0>{
      \draw[history-line] (-.33,-.33) -- ++(.33,.33) -- ++(2,0) coordinate (@2)++(1,0) -- ++(0.5,0) coordinate (@l) -- ++(1,3) -- ++(7.5,0) node[above left,gray] {\vphantom{y}Deductive Methods};
      \draw[history-line] (@l) -- ++(1,1) -- ++(7.5,0) node[above left,gray] {Model Checking};
      \draw[history-line] (@l) -- ++(1,-1) -- ++(7.5,0) node[above left,gray] {Symbolic Execution};
      \draw[history-line] (@l) -- ++(1,-3) -- ++(7.5,0) node[above left,gray] {Abstract Interpretation};
   }
   \only<6-|handout:2->{
      \draw[history-line] (-.33,-.33) -- ++(.33,.33) -- ++(.5,0) coordinate (@2)++(1,0) -- ++(0.5,0) coordinate (@l) -- ++(1,3) -- ++(9,0) node[above left,gray] {\vphantom{y}Deductive Methods};
      \draw[history-line] (@l) -- ++(1,1) -- ++(9,0) node[above left,gray] {Model Checking};
      \draw[history-line] (@l) -- ++(1,-1) -- ++(9,0) node[above left,gray] {Symbolic Execution};
      \draw[history-line] (@l) -- ++(1,-3) -- ++(9,0) node[above left,gray] {Abstract Interpretation};
   }
   \draw[history-line skip] (@2)++(.15,0) -- ++(.85,0);
   \begin{onlyenv}<-5|handout:1>
\pause
   \draw[history-event] (.5,0) -- ++(.25,2) -- ++(.25,0) \historybox{right}{1949}{First Checks}{\citeauthor*{turing1989checking}~\cite{turing1989checking}};
\pause
   \draw[history-event] (1,0) -- ++(.25,1.15) -- ++(.25,0) \historybox{right}{1953}{Rice Theorem}{Non-trivial Properties\\are undecidable~\cite{rice1953classes}};
\pause
   \draw[history-event] (1.5,0) -- ++(.25,-.45) -- ++(.25,0) \historybox{right}{1967\,\&\,69}{Logical Foundation}{\citeauthor*{floyd1967assigning}~\cite{floyd1967assigning}, \citeauthor*{DBLP:journals/cacm/Hoare69}~\cite{DBLP:journals/cacm/Hoare69}\\But: No Automation};
   \end{onlyenv}
% DBLP:conf/cade/OwreRS92
\begin{onlyenv}<7-|handout:2->
   \draw[history-event] (3.5,3) -- ++(.25,1.45) -- ++(.25,0) \historybox{right}{1992}{Theorem Prover}{PVS, \citeauthor*{DBLP:conf/cade/OwreRS92}~\cite{DBLP:conf/cade/OwreRS92}};
   \draw[history-event] (4.5,3) -- ++(.25,.65) -- ++(.25,0) \historybox{right}{2004}{Proof Asisstant}{Coq, \citeauthor*{DBLP:series/txtcs/BertotC04}~\cite{DBLP:series/txtcs/BertotC04}}; % isabelle, agda, ...
\end{onlyenv}
   \onslide<1->
\end{tikzpicture}
\begin{tikzpicture}[overlay,remember picture]
   \node[above right,gray,yshift=3.5mm,font=\tiny,text width=.9\paperwidth] at (current page.south west) {Based on the amazing \citetitle{DBLP:journals/ftpl/Mine17} by \citeauthor{DBLP:journals/ftpl/Mine17}~\cite{DBLP:journals/ftpl/Mine17} and \href{https://web.archive.org/web/20241208213653/https://www.di.ens.fr/~cousot/AI/}{https://www.di.ens.fr/~cousot/AI/}};
\end{tikzpicture}
\end{frame}

\begin{frame}{\insertsubsection}
    Content of first slide
    [TODO: timeline] % https://www.youtube.com/watch?v=IBlfJerAcRw&t=2624s
\end{frame}

\section{The How}

\subsection{Terminology}

\begin{frame}
   What is a Property? Set basis Poset etc. 
   I have to abstract! 
   Galois, Semantics Principles of Abstract Interpretation book
\end{frame}

\subsection{Sign Domain}

\begin{frame}
   
\end{frame}

\subsection{Numerical Domain}

\begin{frame}
   
\end{frame}

\begin{frame}[allowframebreaks]{References}
   \printbibliography
\end{frame}

\end{document}