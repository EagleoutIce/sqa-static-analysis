\documentclass{article}

\usepackage[active,tightpage]{preview}

\usepackage{lmodern,amssymb}
\usepackage[sfdefault]{FiraSans}
\usepackage{mathastext}
\usepackage{tikz}

\def\PreviewBorder{2cm}

\definecolor{red}{HTML}{A32638}
\definecolor{green}{HTML}{56AA1C}
\definecolor{blue}{HTML}{26547C}
\definecolor{orange}{HTML}{DF6D07}
\definecolor{accent}{HTML}{A9A28D}

\usetikzlibrary{arrows.meta,decorations.pathmorphing,fit,decorations.pathreplacing,backgrounds,matrix}
\pgfdeclarelayer{foreground}
\pgfsetlayers{background,main,foreground}
\usepackage{fontawesome}

\begin{document}
\preview
\pgfmathsetseed{42}%
\begin{tikzpicture}[line cap=round]
   \pgfonlayer{foreground}
   \draw[Kite-Kite,very thick] (0,3.5) node[below right,yshift=1mm] {\(x(t)\)} |- (8,0) node[above left] {\(t\)}; % time vs. x at tat time
   \endpgfonlayer
   \colorlet{@}{gray}
   \draw[very thick,@] (0,1) plot [smooth] coordinates {(0,1) (1,2) (2,1) (3,2) (4,1) (5,2) (6,1) (7,2) (8,1)}; % x(t)
   \draw[very thick,@] (0,0) plot [smooth] coordinates {(0,0) (1,1) (2,2) (3,2) (4,2.5) (5,2.5) (6,.5) (7,.6) (8,.6)}; % x(t)
   \draw[very thick,@] (0,1.5) plot [smooth] coordinates {(0,1.5) (1,2) (2,2.5) (3,2) (4,2) (5,2.5) (6,2.5) (7,2) (8,2.5)}; % x(t)
      \foreach \i in {0,...,5} {
         \pgfmathsetmacro{\randA}{rnd*0.33}
         \pgfmathsetmacro{\randB}{rand*0.5}
         \pgfmathsetmacro{\randC}{rand*0.4}
         \draw[gray] (0,1.5+\randA) plot [smooth] coordinates {(0,1.5+\randA) (1,2-\randB) (2,2.5-\randA) (3,2-\randB) (4,2+\randA) (5,2.5) (6,2.5+\randA) (7,2-\randA) (8,2.5+\randB)} node[inner sep=0pt] (a-\i) {};
         \draw[gray] (0,0+\randA) plot [smooth] coordinates {(0,0+\randA) (1,1-\randB) (2,2-\randB) (3,2+\randC) (4,2.5-\randA) (5,2.5-\randB) (6,.5+\randC) (7,.6+\randB) (8,.6+\randC)} node[inner sep=0pt] (b-\i) {};
         \draw[gray] (0,1+\randB) plot [smooth] coordinates {(0,1-\randC) (1,2-\randB) (2,1+\randB) (3,2-\randA) (4,1+\randA) (5,2-\randB) (6,1) (7,2-\randC) (8,1+\randA)} node[inner sep=0pt] (c-\i) {};
      }
   % fit to all nodes to get the bounding box
   \node[fit=(a-0) (a-1) (a-2) (a-3) (a-4) (a-5) (b-0) (b-1) (b-2) (b-3) (b-4) (b-5) (c-0) (c-1) (c-2) (c-3) (c-4) (c-5),inner sep=0pt] (big-ghost) {~};
      % \draw[decorate,thick,decoration={brace,amplitude=5pt,raise=2pt},gray] (big-ghost.north east) -- (big-ghost.south east);
   \pgfonlayer{background}
   \pgfinterruptboundingbox
   \fill[red,opacity=.175,even odd rule] plot [smooth] coordinates {(0,0) (1,0.4) (2,0.5) (3,1) (4,.8) (5,1) (6,.1) (7,0.2) (8.03,.2) (8.03,3) (7,2.8) (6,3) (5,2.95) (4,2.85) (3,2.75) (2,2.65) (1,2.5) (0,2) } -- cycle (6,1.85) circle[radius=4mm]; 
   \endpgfinterruptboundingbox
      \node (@b1) at (6,1.85) {\small\faBug};
      \node (@b2) at (3,.35) {\small\faBug};
      \node (@b3) at (7,2.5) {\small\faBug};
      \node[above left=-1mm,green] at(@b2.south east) {\scriptsize\faCheck};
      \node[above left=-1mm,green] at(@b1.south east) {\scriptsize\faCheck};
      \node[above left=-1mm,yshift=1pt,orange] at(@b3.south east) {\scriptsize\faQuestion};
   \endpgfonlayer
   \path[use as bounding box] (0,0) rectangle (8,3.5);
\end{tikzpicture}
\endpreview
\end{document}